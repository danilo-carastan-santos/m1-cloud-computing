\documentclass[xcolor={dvipsnames}]{beamer}
\mode<presentation>{\usetheme{boxes}}
\usecolortheme{default}
\setbeamertemplate{navigation symbols}{}%remove navigation symbols

\setbeamerfont{frametitle}{size=\Large}

%\setbeamercolor{structure}{fg=beamer@blendedblue}



\setbeamertemplate{bibliography item}{\insertbiblabel}

\setbeamercolor{bibliography entry author}{fg=black}
\setbeamercolor{bibliography entry title}{fg=black} 
\setbeamercolor{bibliography entry location}{fg=black} 
\setbeamercolor{bibliography entry note}{fg=black}  

\usepackage[style=numeric,sorting=ydnt,maxnames=1,defernumbers=true, firstinits=true]{biblatex}
\renewbibmacro{in:}{}
\ExecuteBibliographyOptions{sorting=ydnt}

%\addbibresource{./jlescSummerSchool_checkpointing.bib}

\makeatother
\setbeamertemplate{footline}
{
  \leavevmode%
  \hbox{%
    %% \begin{beamercolorbox}[wd=.2\paperwidth,ht=2.25ex,dp=1ex]{date in head/foot}%
    %%   \usebeamerfont{date in foot}
    %% \end{beamercolorbox}%
  %%   \begin{beamercolorbox}[wd=.6\paperwidth,ht=2.25ex,dp=1ex, center]{date in head/foot}%
  %%     \usebeamerfont{date in foot}\insertshortdate
  %% \end{beamercolorbox}%
  \begin{beamercolorbox}[wd=\paperwidth,ht=2.25ex,dp=1ex]{date in head/foot}%
    \usebeamerfont{date in foot}\hfill
    {\scriptsize\insertframenumber{}}\hspace*{2ex}
  \end{beamercolorbox}}%
  \vskip0pt%
}
\makeatletter



\usepackage[utf8]{inputenc}
%\usepackage[T1]{fontenc}
%\usepackage[francais]{babel}
\usepackage{hyperref}
\usepackage{url}
\usepackage{pifont}
\usepackage{changepage}
\usepackage{listings}
\usepackage{fancyvrb}
\usepackage{multirow}
\usepackage{tabu} 
\usepackage{colortbl}

\usepackage{marvosym}

\usepackage{eurosym}

\usepackage{pdfpages}
\setbeamercolor{background canvas}{bg=}


\usepackage{perpage} %the perpage package
\MakePerPage{footnote}



\definecolor{beamer@blendedblue}{RGB}{0,102,204}

\definecolor{beamer@lightgray}{RGB}{238,238,224}


\definecolor{itemorange}{RGB}{255,114,0}


\definecolor{myorange}{RGB}{255,103,0}


\setbeamertemplate{itemize items}[circle]
\setbeamertemplate{itemize subitem}[triangle]


\setbeamercolor{itemize item}{fg=beamer@blendedblue}
\setbeamercolor{itemize subitem}{fg=gray}


\newcommand<>{\Blue}[1]{{\color#2{beamer@blendedblue}#1}}
%\newcommand<>{\Orange}[1]{{\color#2{BurntOrange}#1}}
\newcommand<>{\Orange}[1]{{\color#2{myorange}#1}}
\newcommand<>{\Alert}[1]{{\Orange{\textbf{#1}}}}


\newcommand{\largeskip}{\vspace{0.6cm}}
\newcommand{\hugeskip}{\vspace{1cm}}



\usepackage{tikz}
\usetikzlibrary{%
decorations.pathreplacing,%
decorations.pathmorphing,%
decorations.shapes,%
decorations.text,%
decorations.markings,%
shapes,%
shapes.callouts,%
shadows,%
arrows,
calc,%
positioning,%
chains,%
backgrounds,%
fit, %
fadings}






\tikzset{
    invisible/.style={opacity=0},
    visible on/.style={alt={#1{}{invisible}}},
    alt/.code args={<#1>#2#3}{%
      \alt<#1>{\pgfkeysalso{#2}}{\pgfkeysalso{#3}} % \pgfkeysalso doesn't change the path
    },
}


\tikzset{
    colornode/.style={
        outer sep=0pt, fill=#1!67, %text height=2ex, text depth=.5ex
    },
    cpu/.style={
        diamond, fill=gray!30, aspect=3, name=CPU#1,
        node contents={$\text{CPU}_{#1}$},
    },
    thread/.style={fill=#1!67,
        minimum width=5ex, minimum height=1.25em},
    tick/.style={very thin},
}



\newcommand{\email}[1]{\href{mailto:#1}{\nolinkurl{#1}}}

\newcommand{\xmark}{\ding{55}}


\AtBeginPart{
  %\frame{\partpage}
  \frame{
    \frametitle{Agenda}
    \small
    \tableofcontents[part=\insertpartnumber,
    sectionstyle=show,
    subsectionstyle=hide,
    subsubsectionstyle=hide]
  }
}

\AtBeginSection[]
{
  \begin{frame}
    \frametitle{Agenda}
    \small
    \tableofcontents[
    sectionstyle=show/shaded,
    subsectionstyle=show/show/hide,
    subsubsectionstyle=hide]
  \end{frame}
}

\AtBeginSubsection[]{
  \mode<presentation>{
    \frame{\tableofcontents[
      sectionstyle=show/hide,
      subsectionstyle=show/shaded/hide,
      subsubsectionstyle=show/show/hide]
    }
  }
}


\date[\the\year]{\the\year}

\newcommand{\shellcmd}[1]{\indent\indent\texttt{\footnotesize\$ #1}}


\title[]{Cloud Computing}
\subtitle{Présentation}


\author[]{\\Danilo Carastan dos Santos
  \\ \vspace{0.5cm} \email{danilo.carastan-dos-santos@univ-grenoble-alpes.fr}}

\definecolor{mybrown}{RGB}{205,133,63}
\definecolor{myblue}{RGB}{28,134,238}

\let\Red=\alert
\newcommand<>{\green}[1]{{\color#2{green!70!black}#1}}
\newcommand<>{\blue}[1]{{\color#2{blue!100!black!100}#1}}
\definecolor{darkgreen}{rgb}{0,0.5,0}

\newcommand\myvdots{{\smash[b]\strut\smash[t]\vdots}}


\usepackage{boxedminipage}
\newenvironment{boitecode}[1]{
    \begin{boxedminipage}{\linewidth}      
%\begin{beamerboxesrounded}[shadow=true,lower=lightex,upper=medex]{#1}
    #1
    \begin{semiverbatim}
}{   \end{semiverbatim}\vspace{-1.5\baselineskip}
    \end{boxedminipage}
%  \end{beamerboxesrounded}
}


\begin{document}


\begingroup
\setbeamercolor{titlelike}{bg=beamer@lightgray, fg=black}
\begin{frame}
\titlepage
\end{frame}

\endgroup



\begin{frame}{Motivations}

  \begin{center}
    \Large{Dans quelques mois, vous allez être sur le marché du travail et
    intégrer un projet en entreprise.}
  \end{center}

  \bigskip
  Pour pouvoir être efficace dans votre nouvel environnement, il vous
  faudra:
  \begin{itemize}
  \item maîtriser un ensemble de \Alert{méthodes} de travail
  \item maîtriser un ensemble de \Alert{technologies}
  \end{itemize}
  
\end{frame}

\begin{frame}{\Alert{Technologie :} Le \textit{Cloud Computing}}

Modèle permettant un accès pratique et à la demande des ressources informatique
configurables (par exemple, réseaux, serveurs, stockage, applications et
services) qui peuvent être rapidement mis à disposition avec un minimum d'effort
de gestion ou d'interaction avec le fournisseur de services.

\vspace{5mm}

\textbf{Modèle largement utilisé :} Facebook, Twitter (X), Salesforce.com, Netflix, \dots
  
\end{frame}

\begin{frame}{Compétences developpées}  

  \begin{footnotesize}
  
  \textbf{Axe 1 : Bases du Cloud}
  \begin{itemize}
    \item Savoir distinguer les types de service Cloud (IaaS, PaaS, SaaS, FaaS)
    \item Comprendre les principes et techniques de virtualisation et conteneurisation  
  \end{itemize}

  \textbf{Axe 2 : Mise à niveau Git}
  \begin{itemize}
  \item Utiliser des outils collaboratifs (Git)
  \item Comprendre les principes CI-CD (\textit{Continuous Intégration \& Continuous Deployment}) 
  \end{itemize}

  \textbf{Axe 3 : Techniques du Cloud} 
  \begin{itemize}
  \item Mettre en place une couche de virtualisation/conteneurisation (VirtualBox, Docker)
  \item Développer des applications Cloud-native\footnote{\scriptsize{Les applications Cloud
  (dites ``cloud-native'') suivent une conception à base de microservices, qui
  permet d'exploiter les fonctionnalités des couches PaaS (par exemple, la
  gestion automatique de l'élasticité pour adapter l'architecture répartie de
  l'application à la charge client).}} via les technologies existantes   
  \end{itemize}
\end{footnotesize}
  
\end{frame}


\begin{frame}{Organisation}

  \begin{block}{Staff}
    \begin{itemize}
    \item Danilo Carastan-Santos
    \item Dimitri Rapacchi
    \item Gérard Forestier    
    \end{itemize}
  \end{block}

  \begin{block}{Organisation}
    \begin{columns}
      \column{0.5\textwidth}
      \begin{itemize}
      \item 9 heures de cours
      \item 18 heures de TP
        \begin{itemize}
        \item Projet: Déployer une application sur un Cloud public
        \end{itemize}
      \end{itemize}

      \column{0.5\textwidth}
      \begin{itemize}
      \item Note:
      \begin{itemize}
      \item 33\% sur le contrôle continu (à définir)
      \item 67\% sur l'examen
      \end{itemize}
      \end{itemize}
    \end{columns}
  \end{block}

\bigskip

  \begin{block}{Site web du cours}
      \begin{itemize}
      \item \url{https://m1-miage-cloud-gitlab-io-4d0541.gitlab.io/}
      \end{itemize}

  \end{block}
  
\end{frame}



\end{document}
